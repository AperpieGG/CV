\documentclass[10pt]{article}
\usepackage[a4paper,bottom = 0.6in,left = 0.75in,right = 0.75in,top = 1cm]{geometry}
\usepackage{graphicx}
\usepackage{amsmath}
\usepackage{array}
\usepackage{enumitem}
\usepackage{wrapfig}
\usepackage{microtype}
\usepackage{titlesec}
\usepackage{graphicx}
\usepackage{booktabs}
\usepackage{url}
\usepackage{enumitem}
\usepackage{palatino}
\usepackage{tabularx}
\fontfamily{SansSerif}
\usepackage{fontawesome5}
\usepackage[hidelinks]{hyperref}
\usepackage{textcomp}
\usepackage{verbatim}
\usepackage{makecell}
\usepackage{pbox}

%for color
\usepackage[usenames, dvipsnames]{color}
\definecolor{myblue}{RGB}{10,0,254}

% Adjusting the margins
\enlargethispage{2\baselineskip} % you can adjust the number based on your content's fit in the page

\newcommand{\xfilll}[2][1ex]{
\dimen0=#2\advance\dimen0 by #1
\leaders\hrule height \dimen0 depth -#1\hfill}
\titleformat{\section}{\large\scshape\raggedright}{}{0em}{}
\renewcommand\labelitemi{\raisebox{0.4ex}{\tiny$\bullet$}}
\renewcommand{\labelitemii}{$\cdot$}
\pagenumbering{gobble}


\usepackage[T1]{fontenc}
\usepackage
%[ansinew]
[utf8]
{inputenc}

\usepackage{color}
\definecolor{mygrey}{gray}{0.99}
\textheight=10in
\raggedbottom

\setlength{\tabcolsep}{0in}
\newcommand{\isep}{-2 pt}
\newcommand{\lsep}{-0.5cm}
\newcommand{\psep}{-0.6cm}
\renewcommand{\labelitemii}{$\circ$}

\pagestyle{empty}
%-----------------------------------------------------------
%Custom commands
\newcommand{\resitem}[1]{\item[{\color[RGB]{10,0,254}$\bullet$}] #1 \vspace{-2pt}}
\newcommand{\resheading}[1]{{\small \colorbox{mygrey}{\begin{minipage}{0.99\textwidth}{\textbf{#1 \vphantom{p\^{E}}}}\end{minipage}}}}
\newcommand{\ressubheading}[3]{
\begin{tabular*}{8.62in}{l @{\extracolsep{\fill}} r}
	\textsc{{\textbf{#1}}} & \textsc{\textit{[#2]}} \\
\end{tabular*}\vspace{-8pt}}


\begin{document}
\begin{table}
    \begin{center}
    {\Huge \scshape Ioannis Apergis} \\ \vspace{1pt}
    Coventry, United Kingdom\\ \vspace{3pt}
    \small \raisebox{-0.1\height}\faPhone\ +30 6971812476 ~ \href{mailto:yourmail@gmail.com}{\raisebox{-0.2\height}\faEnvelope\  \underline{g.aperghc@gmail.com}} ~ 
    \href{https://www.linkedin.com/in/iapergis2401/}{\raisebox{-0.2\height}\faLinkedin\ \underline{I.Apergis}}  ~
    \href{https://github.com/AperpieGG}{\raisebox{-0.2\height}\faGithub\ \underline{AperPieGG}} ~   \href{https://warwick.ac.uk/fac/sci/physics/research/astro/people/ioannisapergis/}{\raisebox{-0.2\height}\faGlobe\ \underline{Warwick}}       
    \vspace{-8pt}
\end{center}
\end{table}    

\setlength{\tabcolsep}{15pt} % decrease the value if the content is 

%%% SHORT SUMMARY %%%

%%%% SCHOLASTIC ACHIEVEMENTS %%%%
% \section*{{\LARGE \color{myblue}Scholastic Achievements}\xfilll[0pt]{0.5pt}}
% \vspace{-15pt}
% \vspace{1.5mm}

% \begin{itemize}[itemsep = -0.70 mm, leftmargin=*]
% \item[{\color[RGB]{10,0,254}$\bullet$}] \noindent Pursuing a \textbf{Minor} in Industrial Design Center(IDC)                         
% \item[{\color[RGB]{10,0,254}$\bullet$}] \noindent Ranked in the top \textbf{3 \%} among \textbf{0.2 million} candidates in JEE Advanced 2017 \hfill {\ \small [April'17]}
% \item[{\color[RGB]{10,0,254}$\bullet$}] \noindent Ranked in the top \textbf{1 \%} among \textbf{1.5 million} candidates in JEE Mains 2017 \hfill {\ \small [June'17]}
% \item[{\color[RGB]{10,0,254}$\bullet$}] \noindent Secured the \textbf{1st position} in intra-state Talent Search Contest among \textbf{0.1 million} students conducted by \textbf{Navodaya Educational Society And Welfare Committee, Bhopal}\hfill {\ \small [July'15]} 
% \item[{\color[RGB]{10,0,254}$\bullet$}] \noindent Awarded \textbf{SBI Scholar} Fellowship which is given to meritorious children of SBI staff \hfill {\ \small [July'17]}

% \end{itemize}
% \vspace{-15pt}
% \vspace{-7pt}

%%%%%%%%%%%%%%%%%%%%%%%% EDUCATION %%%%%%%%%%%%%%%%%%%%%%%%
\section*{{\LARGE \color{myblue}Education}\xfilll[0pt]{0.5pt}}
\vspace{-15pt}
\vspace{1.5mm}

%\begin{itemize}[itemsep = -0.75 mm, leftmargin=*]
\textbf{\large PhD student}| \normalfont{University of Warwick} \hfill { \small Oct 2022 -- Oct 2026}
\\[-0.5cm]
 
	\begin{itemize}[itemsep = -0.75 mm, leftmargin=*]
	\item[{\color[RGB]{10,0,254}$\bullet$}] \noindent Industrial CASE PhD student at Astronomy and Astrophysics Group
        \end{itemize} 
\textbf{\large Master of Physics honours}| \normalfont{University of Kent} \hfill { \small Sep 2018 -- Jul 2022}
\\[-0.5cm]

	\begin{itemize}[itemsep = -0.75 mm, leftmargin=*]
	\item[{\color[RGB]{10,0,254}$\bullet$}] \noindent First Class Honours in  Astronomy Space Science and Astrophysics
        \end{itemize} 
\vspace{-15pt}

%%%%%%%%%%%%%%%%%%%%%%%%% PROJECTS %%%%%%%%%%%%%%%%%%%%%%%%%
\section*{{ \LARGE \color{myblue} Projects}\xfilll[0pt]{0.5pt}}
\vspace{-13pt}
\vspace{1.5mm}

\textbf{\large Precise Photometry with the new generation sCMOS Camera}| \normalfont{PhD} \hfill {\ \small Ongoing}
\\[0.05cm]
\emph{University of Warwick} \hfill {\ \small (Expected by Oct '26)} 
\\[-0.6cm]
\begin{itemize}[itemsep = -0.75 mm, leftmargin=*]
    \item[{\color[RGB]{10,0,254}$\bullet$}] \noindent Understanding and characterise the new sCMOS camera. Developing and executing a control system code in python, record and analyse data. Observations of exoplanets using the
    NGTS telescopes at Paranal observatory using sCMOS and CCDs cameras. 
\end{itemize} 	
\textbf{\large Direct detection of Asteroidal YORP effect in 1950 DA}| \normalfont{MPhys} \hfill {\ \small Sep 2021 -- July 2022}
\\[0.05cm]
\emph{University of Kent}  
\\[-0.6cm]
\begin{itemize}[itemsep = -0.75 mm, leftmargin=*]
    \item[{\color[RGB]{10,0,254}$\bullet$}] \noindent Modelling small Solar
    system bodies and analyse their physical properties and state. Photometric analysis of the asteroid and data
    reduction from CCDs from large observational facilities.
\end{itemize} 	

 
%%%%%%%%%%%%%%%%%%%%%%%% SKILLS %%%%%%%%%%%%%%%%%%%%%%%%
\section*{{\LARGE \color{myblue}Technical Skills}\xfilll[0pt]{0.5pt}}

\vspace{-22pt}
\setlength{\tabcolsep}{9pt}
\begin{table}[h]

\begin{tabular}{lllll}

\textbf{Programming Languages} & Python, Matlab \\ 
\textbf{Astronomy software} &  IRAF, AIP4WIN, Aladin sky Atlas, SAOImageDS9 \\ 
\textbf{Operating systems} & MacOS, Linux, Windows \\ 
\textbf{Tools and Environments} & Git, GitHub, PyCharm, VSCode

\end{tabular}
\end{table}
% \fi 

\vspace{-20pt}
% \fi 

%%%%%%%%%%%%%%%%%%% PROFESSIONAL ACTIVITIES %%%%%%%%%%%%%%%%%%%%%%%
\section*{{ \LARGE \color{myblue} Working Activities}\xfilll[0pt]{0.5pt}}
\vspace{-13pt}
\vspace{1.5mm}

\textbf{\large Studentship Physicist}| \normalfont{Andor Technology, Belfast} \hfill {\ \small May 2023 -- August 2023}
\\[-0.6cm]
\begin{itemize}[itemsep = -0.75 mm, leftmargin=*]
    \item[{\color[RGB]{10,0,254}$\bullet$}] \noindent Work with the R\&D team to optimise and improve technical aspects of the
    camera. Also interact with Andor’s Engineering and Quality Control Team to learn and execute methods in effort to
    the characterise the camera according to EMVA-1288 standards. The work continued online from Warwick. I intend to spent 9 months in total at Andor as part of the industrial CASE PhD.
	\end{itemize} 		
\textbf{\large Summer Student}| \normalfont{University of Kent, Canterbury} \hfill {\ \small June 2022 -- August 2022}
\\[-0.6cm]
\begin{itemize}[itemsep = -0.75 mm, leftmargin=*]
    \item[{\color[RGB]{10,0,254}$\bullet$}] \noindent This research is a continuation of the master’s project. In this work, extra data of previous years of the asteroid
    1950 DA was analysed. Also scanning for different pole solutions of the asteroid’s shape.
\end{itemize} 	

\vspace{-15pt}


%%%%%%%%%%%%%%%%%%%%%%%% PUBLICATIONS %%%%%%%%%%%%%%%%%%%%%%%%%%%%
\section*{{\LARGE \color{myblue}Publications}\xfilll[0pt]{0.5pt}}
\vspace{-8pt}

\textbf{\large{Lead Author}}
\\[-0.6cm]
\begin{itemize}[itemsep = -0.75 mm, leftmargin=*]
    \item[{\color[RGB]{10,0,254}$\bullet$}] \noindent \textit{"Characterisation of the sCMOS Marana camera for high precision Photometry"}.  Work complete, draft paper in preparation.
    \item[{\color[RGB]{10,0,254}$\bullet$}] \noindent \textit{"On-sky Precise Photometry with a fast readout sCMOS camera"}.  Work to be complete over the next 6 months.
\end{itemize}
\vspace{-2pt}
\textbf{\large{Co-Author}}  
\\[-0.6cm]
\begin{itemize}[itemsep = -0.75 mm, leftmargin=*]
    \item[{\color[RGB]{10,0,254}$\bullet$}] \noindent \textit{Henderson, B. et al. (2023),  "TOI-2490b- The most eccentric brown dwarf transiting in the brown dwarf desert"}, (in prep).
    \item[{\color[RGB]{10,0,254}$\bullet$}] \noindent \textit{Hawthorn, F. et al. (2023), "TESS Duotransit Candidates from the Southern Ecliptic Hemisphere"}, (submitted in MNRAS).
    \item[{\color[RGB]{10,0,254}$\bullet$}] \noindent \textit{Gill, S. et al. (2023), "TOI-2447b / NGTS-29b: a 69-day Saturn around a Solar analogue"} (submitted in MNRAS).
    \item[{\color[RGB]{10,0,254}$\bullet$}] \noindent \textit{Anderson D. R. et al. (2023), "NGTS-11 c: a Neptune-mass planet in a transiting 12.8-day orbit interior to NGTS-11 b"} (submitted in A\&A).
\end{itemize}

%%%%%%%%%%%%%%%%%%%%%%%% PROPOSALS %%%%%%%%%%%%%%%%%%%%%%%%%%%%

\section*{{\LARGE \color{myblue}Proposals}\xfilll[0pt]{0.5pt}}
\vspace{-8pt}

\textbf{\large{Co-Investigator}}
\\[-0.6cm]
\begin{itemize}[itemsep = -0.75 mm, leftmargin=*]
    \item[{\color[RGB]{10,0,254}$\bullet$}] \noindent Anderson D. R. et al. (2023), 112.25TM: \textit{”Characterizing two multi-planet systems: TIC 193096383 and TIC 398986735"}.
\end{itemize}


%%%%%%%%%%%%%%%%%%%% CONFERENCES AND MEETINGS %%%%%%%%%%%%%%%%%%%%
\section*{{\LARGE \color{myblue}Conferences and Meetings}\xfilll[0pt]{0.5pt}}
\vspace{-15pt}
\vspace{1.5mm}

%\begin{itemize}[itemsep = -0.75 mm, leftmargin=*]
\textbf{\large NGTS Consortium Meeting}| \normalfont{Queen's University Belfast} \hfill { \small April 2023}
\\[-0.5cm]

\begin{itemize}[itemsep = -0.75 mm, leftmargin=*]
    \item[{\color[RGB]{10,0,254}$\bullet$}] \noindent Presented my work on the characterisation of the sCMOS camera
\end{itemize} 
\textbf{\large NGTS Consortium Meeting}| \normalfont{University of Warwick (Online)} \hfill { \small Sep 2023}
\\[-0.5cm]

\begin{itemize}[itemsep = -0.75 mm, leftmargin=*]
    \item[{\color[RGB]{10,0,254}$\bullet$}] \noindent Presented my work and involvement with Andor
\end{itemize} 
\textbf{\large GNOSIS Annual Conference}| \normalfont{University of Warwick} \hfill { \small Nov 2022}
\\[-0.5cm]

\begin{itemize}[itemsep = -0.75 mm, leftmargin=*]
    \item[{\color[RGB]{10,0,254}$\bullet$}] \noindent Space Sustainability for the Next Decade (and Beyond)
\end{itemize} 
\textbf{\large Epistemic Insight}| \normalfont{Canterbury Christ Church University} \hfill { \small July 2023}
\\[-0.5cm]

\begin{itemize}[itemsep = -0.75 mm, leftmargin=*]
    \item[{\color[RGB]{10,0,254}$\bullet$}] \noindent Astronomy with AI Summer school
\end{itemize} 
\textbf{\large Writing Better Research Papers and Proposals}| \normalfont{Berlin (Online)} \hfill { \small Nov 2023}
\\[-0.5cm]

\begin{itemize}[itemsep = -0.75 mm, leftmargin=*]
    \item[{\color[RGB]{10,0,254}$\bullet$}] \noindent Online Workshop on Writing better research papers and proposals
\end{itemize}         
\textbf{\large TESS Science Team Meeting}| \normalfont{MIT (Online)} \hfill { \small Nov 2022}
\\[-0.5cm]

\begin{itemize}[itemsep = -0.75 mm, leftmargin=*]
    \item[{\color[RGB]{10,0,254}$\bullet$}] \noindent TESS Science group meeting about the technical aspects and the next steps of the mission
\end{itemize}         
\vspace{-15pt}


%%%%%%%%%%%%%%%%%%%%%%%% SEMINARS %%%%%%%%%%%%%%%%%%%%%%%%%%%%
\section*{{\LARGE \color{myblue}Seminars}\xfilll[0pt]{0.5pt}}
\vspace{-8pt}

\textbf{\large{Centre of Astrophysics and Planetary Science (CAPS)}}
\\[-0.6cm]
\begin{itemize}[itemsep = -0.75 mm, leftmargin=*]
    \item[{\color[RGB]{10,0,254}$\bullet$}] \noindent Participating in wide range of seminars about astrophysics during my UG course.
\end{itemize}
\vspace{-2pt}
\textbf{\large{Warwick Astronomy Seminars}}  
\\[-0.6cm]
\begin{itemize}[itemsep = -0.75 mm, leftmargin=*]
    \item[{\color[RGB]{10,0,254}$\bullet$}] \noindent Astronomy seminars of specialists in the fields. 
\end{itemize}
\vspace{-2pt}
\textbf{\large{Warwick Physics Colloquial}}  
\\[-0.6cm]
\begin{itemize}[itemsep = -0.75 mm, leftmargin=*]
    \item[{\color[RGB]{10,0,254}$\bullet$}] Participate in a variety of broad area of physics seminars from high profile.
    speakers.
\end{itemize}
\vspace{-2pt}
\textbf{\large{Warwick Exoplanets Meeting group}}  
\\[-0.6cm]
\begin{itemize}[itemsep = -0.75 mm, leftmargin=*]
    \item[{\color[RGB]{10,0,254}$\bullet$}] I gave a talk about the characterisation of sCMOS Marana camera and my work in Andor at Belfast.
\end{itemize}
\textbf{\large ToF and FMCW LiDARs}| \normalfont{London (Online)} \hfill { \small Nov 2023}
\\[-0.5cm]

\begin{itemize}[itemsep = -0.75 mm, leftmargin=*]
    \item[{\color[RGB]{10,0,254}$\bullet$}] \noindent A SPIE.ONLINE webinar about ToF and FMCW LiDARs: a closer look at the principles of operation and technical challenges 
\end{itemize}
\textbf{\large Observing NEAs using telescopes located in South Africa}| \normalfont{South Aftica (Online)} \hfill { \small Nov 2023}
\\[-0.5cm]

\begin{itemize}[itemsep = -0.75 mm, leftmargin=*]
    \item[{\color[RGB]{10,0,254}$\bullet$}] \noindent  Andor webinar about the follow-up characterisation of near-Earth asteroids (NEAs) using telescopes located in Sutherland, South Africa
\end{itemize}

%%%%%%%%%%%%%%%%%%%%%%%% TEACHING %%%%%%%%%%%%%%%%%%%%%%%%%%%%
\section*{{\LARGE \color{myblue}Teaching and Demonstrating}\xfilll[0pt]{0.5pt}}
\vspace{-15pt}
\vspace{1.5mm}

\begin{itemize}[itemsep = -0.75 mm, leftmargin=*]
\item[{\color[RGB]{10,0,254}$\bullet$}] \noindent {\href{https://courses.warwick.ac.uk/modules/2022/PX283-30}\faLink } \textbf{PX283 Laboratory Demonstrator}: Demonstrate and assess A1-Simulations of Planetary Systems lab for $\mathrm{2^{nd}}$ year Physics and Astrophysics students. 
\item[{\color[RGB]{10,0,254}$\bullet$}] \noindent {\href{https://courses.warwick.ac.uk/modules/2022/PX451-15}\faLink } \textbf{PX451 Laboratory Demonstrator}: Demonstrate and assess the $\mathrm{3^{rd}}$ year Astrophysics laboratory using The Marsh Observatory for astronomical techniques and data analysis. This includes observational scheduling and monitoring. 
\end{itemize}


%%%%%%%%%%%%%%%%%%%%%%%% OUTREACH %%%%%%%%%%%%%%%%%%%%%%%%%%%%
 \section*{{\LARGE \color{myblue}Extracurricular and Engagement}\xfilll[0pt]{0.5pt}}
\vspace{-15pt}
\vspace{1.5mm}
\textbf{\large La Palma Observatory} \hfill {\ \small Nov 2023}
\\[-0.6cm]
\begin{itemize}[itemsep = -0.75 mm, leftmargin=*]
    \item[{\color[RGB]{10,0,254}$\bullet$}] \noindent I took part in a technical mission to the telescopes at La Palma observatory to gain valuable insights into technical aspects of telescope operations. 
\end{itemize} 
\textbf{\large{GNOSIS Annual Conference}} \hfill {\ \small Nov 2022}
\\[-0.6cm]
\begin{itemize}[itemsep = -0.75 mm, leftmargin=*]
    \item[{\color[RGB]{10,0,254}$\bullet$}] \noindent I had the chance to interact with
    visiting schools to the university of Warwick and give astronomy talks to children as part of the Astronomy and
    Astrophysics group.	
\end{itemize}
\textbf{\large{{\href{https://codingwithsophie.warwick.ac.uk/}\faLink } Coding With Sophie}} \hfill {\ \small Nov 2023}
\\[-0.6cm]
\begin{itemize}[itemsep = -0.75 mm, leftmargin=*]
    \item[{\color[RGB]{10,0,254}$\bullet$}] \noindent An astrophysicist designed program introducing coding to kids.
\end{itemize}
 \textbf{\large{Warwick Astronomy Society}} \hfill {\ \small Nov 2023}
\\[-0.6cm]
\begin{itemize}[itemsep = -0.75 mm, leftmargin=*]
    \item[{\color[RGB]{10,0,254}$\bullet$}] \noindent Gave a talk on Warwick's Astronomy Society.
\end{itemize}
 \textbf{\large{PLANCKS UK and IRELAND}} \hfill {\ \small Feb 2022}
\\[-0.6cm]
\begin{itemize}[itemsep = -0.75 mm, leftmargin=*]
    \item[{\color[RGB]{10,0,254}$\bullet$}] \noindent Physics League Across Numerous Countries for Kick-Ass Students” is a
    theoretical physics competition for UG. The competition consists of a series of physics problems, covering topics in physics.
\end{itemize}
\textbf{\large{Epistemic Insight}} \hfill {\ \small Jul 2023}
\\[-0.6cm]
\begin{itemize}[itemsep = -0.75 mm, leftmargin=*]
    \item[{\color[RGB]{10,0,254}$\bullet$}] \noindent Gave an interview about the Astronomy With AI Summer school. 
\end{itemize}
\textbf{\large{First Aid at Work}} \hfill {\ \small Nov 2022 -- 2025}
\\[-0.6cm]
\begin{itemize}[itemsep = -0.75 mm, leftmargin=*]
    \item[{\color[RGB]{10,0,254}$\bullet$}] \noindent  I have received a first aid certificate in First Aid at Work course under the Health Safety (First Aid) Regulations 1981 Including Defibrillator Training.
\end{itemize}
\textbf{\large{Royal Astronomical Society}} \hfill {\ \small Mar 2023}
\\[-0.6cm]
\begin{itemize}[itemsep = -0.75 mm, leftmargin=*]
    \item[{\color[RGB]{10,0,254}$\bullet$}] \noindent I am a Fellow of the Royal Astronomical Society (FRAS).
\end{itemize}

%%%%%%%%%%%%%%%%%%%%%%%% LANGUAGES %%%%%%%%%%%%%%%%%%%%%%%%%%%%
\section*{{\LARGE \color{myblue}Languages}\xfilll[0pt]{0.5pt}}
\vspace{-15pt}
\vspace{1.5mm}

\begin{itemize}[itemsep = -0.75 mm, leftmargin=*]
\item[{\color[RGB]{10,0,254}$\bullet$}] \noindent Greek (Native).
\item[{\color[RGB]{10,0,254}$\bullet$}] \noindent English (IELTS).
\end{itemize}

\end{document}
